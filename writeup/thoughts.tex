\documentclass{article}
\usepackage[colorlinks=true, linkcolor=blue, citecolor=black, urlcolor=black]{hyperref}
\usepackage[style=trad-abbrv]{biblatex}
\usepackage{amsmath,amsthm,amssymb,dsfont}

\newcommand{\PairDecomp}{\mathcal{W}}

\title{The Strong Lottery Ticket Hypothesis Extended}
\author{Addison Hanrattie}

\newtheorem{thm}{Theorem}[section]
\newtheorem{lem}[thm]{Lemma}
\DeclareMathOperator{\dist}{dist}

\begin{document}
\section{Observations}
A single bulb can have a effect on the solution of max $O(nr)$ where $r$ is the radius of the bulb and $n$ is the number of points inside. 

%If we target the wrong initial inner TSP point we incur a max penalty of $4r$ the same goes for a poorly chosen exit point. Thus so long as we accurately solve the inner the rest of the way we incur a penalty of at most $8r$.

\section{Definitions}
A \textbf{Pair Decomposition} is a of $P$ is a set of pairs $$\PairDecomp = \left\{ \{A_1, B_1\}, \ldots, \{A_k,B_k\} \right\}$$
such that (I) $\forall i; A_i,B_i \subset P$, (II) $\forall i; A_i \cap B_i = \emptyset$, and (III) $\bigcup_{i=1}^k A_i \otimes B_i = \binom{P}{2} = P \otimes P$.

\section{Some Proved Ideas}
\begin{thm}
Let $A$ and $B$ be two finite sets of points that are s-well-separated, let $x,p \in A$, and let $y,q \in B$. Then the following inequalities hold:

\begin{align*}
    |xy| &\leq (1 + 2/s) \cdot |xq| \\
    |xy| &\leq (1 + 4/s) \cdot |pq| \\
    |px| &\leq (2/s) \cdot |pq| \\
\end{align*}
\end{thm}

% 8 points of interest 
% a b c D e f g H i j k
% k j i H g f e D c b a

% D abc-efg-ijk H   AD HK
% D kji-gfe-cba H   DK AH

% Defg-ijk abc H    AK CH
% Defg-ijk cba H    CK AH

% D abc-efg kjiH    AD GK
% D gfe-cba kjiH    DG AK

% Dcba gfe kjiH     AG EK
% Dcba efg kjiH     AE GK

% Defg abc kjiH     AG CK
% Defg cba kjiH     CG AK

% D ijk abc-efgH    DI AK
% D kji abc-efgH    DK AI

% Dcba efg-ijk H    AE HK
% Dcba kji-gfe H    AK GH

% Dcba kji efgH     AK EI
% Dcba ijk efgH     AI EK






\begin{thm}
Within a ball of diameter $d$ within a metric space the difference in shortest hamiltonian paths between any two points is at most $2d$.
\end{thm}
\begin{proof}
    Fix one pair of endpoints $(a,b)$ and let $P_{a,b}$ be the optimal Hamiltonian path from $a$ to $b$ with length denoted $L(a,b)$.
    Next consider another pair of endpoints $(c,d)$ we can then build a hamiltonian path from $c$ to $d$ using $P_{a,b}$.
    \begin{enumerate}
        \item Start at $c$
        \item Travel from $c$ to $a$ for a cost of $\dist(c,a) \leq d$
        \item Follow the path $P_{a,b}$ from $a$ to $b$ for a cost of $c \leq L(a,b)$. (shortcutting $c,d$)
        \item Travel from $b$ to $d$ for a cost of $\dist(b,d) \leq d$
    \end{enumerate}
    %WLOG assume that $c$ occurs before $d$ on the path $P_{a,b}$. 
    %Then let $x$ be the point which precedes $c$ on the path and let $x'$ be the point that immediately follows. 
    %In the same way define $y$ and $y'$ for point $d$.
    %Since the triangle inequality must be obeyed we know 
    Since the space obeys the triangle inequality we know that during step 3 we can only decrease the cost by shortcutting. Thus the total cost of this path is at most $L(a,b) + 2d$. Furthermore we know that this cost must ceil whatever the optimal hamiltonian path from $c$ to $d$ is. Thus we have shown that $L(c,d) \leq L(a,b) + 2d$. By symmetry of the argument we can also show that $L(a,b) \leq L(c,d) + 2d$. Combining these two inequalities we have that $\forall a,b,c,d;|L(a,b) - L(c,d)| \leq 2d$ as desired. 
\end{proof}

\appendix
\section{Outdated proofs}


\end{document}