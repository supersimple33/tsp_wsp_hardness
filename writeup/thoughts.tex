\documentclass{article}
\usepackage[colorlinks=true, linkcolor=blue, citecolor=black, urlcolor=black]{hyperref}
\usepackage{amsmath,amsthm,amssymb,dsfont}
\usepackage{thmtools}
\usepackage{algorithm,algpseudocodex}
\usepackage{cleveref}
\usepackage[section]{placeins}
%\usepackage{draftwatermark}
%\SetWatermarkScale{8}
%\SetWatermarkText{\normalfont{DRAFT}}

\newcommand{\PairDecomp}{\mathcal{W}}

\title{TSP and WSP Thoughts and Ideas}
\author{Addison Hanrattie}

\newtheorem{thm}{Theorem}[section]
\newtheorem{lem}[thm]{Lemma}
\newtheorem{corollary}{Corollary}[thm]
\newtheorem*{rem*}{Remark}
\theoremstyle{definition}
\newtheorem{defn}[thm]{Definition}


\newcommand{\BigLength}{\mathcal{L}}
\newcommand{\Cover}{\mathcal{C}}

\newcommand{\dist}{\delta}
%\DeclareMathOperator{\dist}{dist}

\DeclareMathOperator{\diam}{diam}

\begin{document}
\maketitle
\tableofcontents

\section{Work Area}
The main theorem I am proving here is Corollary~\ref{cor:wsp_tsp} and Theorem~\ref{thm:bin_wspd_tour_structure}.

\section{Definitions}
\subsection{Background Definitions}
\begin{defn}
    A \textbf{Pair Decomposition} is a of $P$ is a set of pairs
    $$\PairDecomp = \left\{ \{A_1, B_1\}, \ldots, \{A_k,B_k\} \right\}$$
    such that (I) $\forall i; A_i,B_i \subset P$, (II) $\forall i; A_i \cap B_i = \emptyset$, and (III) $\bigcup_{i=1}^k A_i \otimes B_i = \binom{P}{2} = P \otimes P$.
\end{defn}

\begin{defn}
    $\dist(Q,R) = \min_{q \in Q, r \in R} |qr|$
\end{defn}

\begin{defn}
    The pair $\{Q,R\}$ is \textbf{$s$-well-separated} if
    $$\max\{\diam(Q), \diam(R)\} \leq \frac{1}{s} \cdot \dist(Q,R)$$
\end{defn}

\subsection{My definitions}

\begin{defn}
    Let $P = A \cup B$ (with $A \cap B = \emptyset$). We say that $A$ and $B$ are \textbf{TSP-Separated} if there exists an optimal TSP tour on $P$ that visits all points in $A$ before any point in $B$ (or vice versa). 
    Furthermore, Let $P = A \cup B \cup C$ (with all three pairwise disjoint). We say that $A$, $B$, and $C$ are \textbf{TSP-Separated} if 
    there exists an optimal TSP tour on $P$ that has exactly 3 or 4 edges connecting the three sets. 
\end{defn}

\begin{defn}
    Let $\{(A_1, B_1), \dots, (A_n, B_n)\} = \PairDecomp$ be a Well Separated Pair Decomposition of the point set $P$.
    We say that a given pair $(A_i, B_i) \in \PairDecomp$ is a \textbf{$k$-binary WSP} if either $A_i \cup B_i = P$ (which we call $1$-binary WSP) or there exists another pair $(A_j, B_j) \in \PairDecomp$ such that $A_i \cup B_i$ is equal to either $A_j$ or $B_j$ and furthermore the pair $(A_j, B_j)$ is a $(k-1)$-binary WSP.
\end{defn}

\begin{defn}
    A Well Separated Pair Decomposition $\PairDecomp$ of the point set $P$ is a \textbf{$k$-binary WSPD} if the union of all $k$-binary WSPs in $\PairDecomp$ is equal to $P$.
\end{defn}

\begin{defn}
    A Well Separated Pair Decomposition $\PairDecomp$ of the point set $P$ is a \textbf{full binary WSPD} if $\forall (A_i, B_i) \in \PairDecomp$ the pair is a $k$-binary WSP for some $k$.
\end{defn}

\section{Background Theorems}
\begin{thm}
Let $A$ and $B$ be two sets of points that are $s$-well-separated, let $p,p' \in A$, and let $q,q' \in B$. 
Then the following inequalities hold:
\begin{align*}
    |pq| &\leq (1 + 1/s) \cdot |pq'| \\
    |pq| &\leq (1 + 2/s) \cdot |p'q'| \\
    |pp'| &\leq (1/s) \cdot |pq| \\
\end{align*}
\end{thm}

\begin{proof}
    Let $A$ and $B$ be two se-ts of points that are $s$-well-separated, let $p,p' \in A$, and let $q,q' \in B$.

    Note that $|qq'| \leq \diam(B) \leq \frac{1}{s} \cdot \dist(A,B) \leq \frac{1}{s} |pq'|$ by the definition of well-separated pairs.
    The first result then follows from the next analysis:
    \begin{align*}
        |pq| &\leq |pq'| + |q'q| \\
        &\leq |pq'| + \frac{1}{s} |pq'| \\
        &= (1 + 1/s) \cdot |pq'|
    \end{align*}

    For the second result we note that $|pp'| \leq \diam(A) \leq \frac{1}{s} \cdot \dist(A,B) \leq \frac{1}{s} |p'q'|$ by the definition of well-separated pairs.
    The second result then follows from the next analysis:
    \begin{align*}
        |pq| &\leq |pp'| + |p'q'| + |q'q| \\
        &\leq |pp'| + |p'q'| + \frac{1}{s} |p'q'| \\
        &\leq \frac{1}{s} |p'q'| + |p'q'| + \frac{1}{s} |p'q'| \\
        &= (1 + 2/s) \cdot |p'q'|
    \end{align*}
    
    Finally for the third result we note that the following is true $|pp'| \leq \diam(A) \leq \frac{1}{s} \cdot \dist(A,B) \leq \frac{1}{s} |pq|$ by the definition of well-separated pairs. The result is immediately apart that $|pp'| \leq (1/s) \cdot |pq|$.
\end{proof}

\section{My Theorems}
\subsection{Hamiltonian Endpoint Distance Bound}

\begin{thm}[You can ignore]
Within a ball of diameter $d$ within a metric space the difference in shortest hamiltonian paths between any two points is at most $2d$.
\end{thm}
\begin{proof}
    Fix one pair of endpoints $(a,b)$ and let $P_{a,b}$ be the optimal Hamiltonian path from $a$ to $b$ with length denoted $L(a,b)$.
    Next consider another pair of endpoints $(c,d)$ we can then build a hamiltonian path from $c$ to $d$ using $P_{a,b}$.
    \begin{enumerate}
        \item Start at $c$
        \item Travel from $c$ to $a$ for a cost of $\dist(c,a) \leq d$
        \item Follow the path $P_{a,b}$ from $a$ to $b$ for a cost of $c \leq L(a,b)$. (shortcutting $c,d$)
        \item Travel from $b$ to $d$ for a cost of $\dist(b,d) \leq d$
    \end{enumerate}
    %WLOG assume that $c$ occurs before $d$ on the path $P_{a,b}$. 
    %Then let $x$ be the point which precedes $c$ on the path and let $x'$ be the point that immediately follows. 
    %In the same way define $y$ and $y'$ for point $d$.
    %Since the triangle inequality must be obeyed we know 
    Since the space obeys the triangle inequality we know that during step 3 we can only decrease the cost by shortcutting. Thus the total cost of this path is at most $L(a,b) + 2d$. Furthermore we know that this cost must ceil whatever the optimal hamiltonian path from $c$ to $d$ is. Thus we have shown that $L(c,d) \leq L(a,b) + 2d$. By symmetry of the argument we can also show that $L(a,b) \leq L(c,d) + 2d$. Combining these two inequalities we have that $\forall a,b,c,d;|L(a,b) - L(c,d)| \leq 2d$ as desired. 
\end{proof}
I haven't been able to prove it either way yet, but I believe the bound can be tightened for this theorem. 

\subsection{Well-Separated Sets are Path/Tour Separated}
\subsubsection{Base Level Analysis}

\begin{thm}\label{thm:wsp_path_ab}
    Suppose we have two sets of points $A$ and $B$ that are $s$-well-separated where $s \geq 1$. Then there exists an optimal hamiltonian path between endpoints $a' \in A$ and $b' \in B$ over $A \cup B$ that visits all points in $A$ before any point in $B$. Furthermore if $s > 1$ then all optimal hamiltonians must visit all points in $A$ before any point in $B$.
\end{thm}
\begin{proof}
    Assume that for some optimal hamiltonian path between endpoints $a' \in A$ and $b' \in B$ over $A \cup B$ it does not visit all points in $A$ before any point in $B$.
    Then it must contain at least three edges crossing between $A$ and $B$.
    Let these edges be $(a_1, b_1), (b_2, a_2), (a_3, b_3)$ ordered as they appear in the path.
    This gives the following partial structure of the path:
    $$a' \leadsto a_1 \to b_1 \leadsto b_2 \to a_2 \leadsto a_3 \to b_3 \leadsto b'$$

    Since the path is optimal we know that it satisfies the 2-optimality condition implying the following:
    $|a_1 b_1| + |a_3 b_3| \leq |a_1 a_3| + |b_1 b_3|$.
    This is to say that the following path must be at least as long as the original path
    $$a' \leadsto a_1 \to a_3 \leadsto a_2 \to b_2 \leadsto b_1 \to b_3 \leadsto b'$$

    Since $A$ and $B$ are $s$-well-separated we know that $\forall i,j; |a_i a_j| \leq \frac{1}{s} |a_i b_i|$. From this we can conclude the following:
    \begin{align*}
        |a_1 b_1| + |a_3 b_3| &\leq |a_1 a_3| + |b_1 b_3| \\
        &\leq \frac{1}{s} |a_1 b_1| + \frac{1}{s} |a_3 b_3| \\
        &= \frac{1}{s} (|a_1 b_1| + |a_3 b_3|)
    \end{align*}
    This implies that $1 \leq \frac{1}{s}$ which in turn implies that $s \leq 1$. 
    If $s > 1$ this is a contradiction and therefore any non-consecutive path cannot be optimal.
    If $s=1$ then we can conclude that $|a_1 b_1| + |a_3 b_3| = |a_1 a_3| + |b_1 b_3|$ which implies that the newly constructed path is also optimal. 
    It follows via induction that any extra loop between $A$ and $B$ can be removed to construct an optimal hamiltonian path that visits all points in $A$ before any point in $B$.
\end{proof}

\begin{thm}\label{thm:wsp_path_aa}
    Suppose we have two sets of points $A$ and $B$ that are $s$-well-separated where $s \geq 1$. Then there exists an optimal hamiltonian path between endpoints $a', a'' \in A$ over $A \cup B$ that visits all points in $B$ consecutively. Furthermore if $s > 1$ then all optimal hamiltonian paths must visit all points in $B$ consecutively.
\end{thm}
\begin{proof}
    Assume that for some optimal hamiltonian path between endpoints $a', a'' \in A$ over $A \cup B$ it does not visit all points in $B$ consecutively.
    Then the path must contain at least four edges crossing between $A$ and $B$.
    Let these edges be $(a_1, b_1), (b_2, a_2), (a_3, b_3), (b_4, a_4)$ ordered as they appear in the path.
    This gives the following partial structure of the path:
    $$a' \leadsto a_1 \to b_1 \leadsto b_2 \to a_2 \leadsto a_3 \to b_3 \leadsto b_4 \to a_4 \leadsto a''$$

    Since the path is optimal we know that it satisfies the 2-optimality condition implying the following:
    $|b_2 a_2| + |b_4 a_4| \leq |b_2 b_4| + |a_2 a_4|$.
    This is to say that the following path must be at least as long as the original path:
    $$a' \leadsto a_1 \to b_1 \leadsto b_2 \to b_4 \leadsto b_3 \to a_3 \leadsto a_2 \to a_4 \leadsto a''$$

    Since $A$ and $B$ are $s$-well-separated we know that $\forall i,j; |a_i a_j| \leq \frac{1}{s} |a_i b_i|$. From this we can conclude the following:
    \begin{align*}
        |b_2 a_2| + |b_4 a_4| &\leq |b_2 b_4| + |a_2 a_4| \\
        &\leq \frac{1}{s} |b_2 a_2| + \frac{1}{s} |b_4 a_4| \\
        &= \frac{1}{s} (|b_2 a_2| + |b_4 a_4|)
    \end{align*}
    This implies that $1 \leq \frac{1}{s}$ which in turn implies that $s \leq 1$.
    If $s > 1$ this is a contradiction and therefore any non-consecutive path cannot be optimal.
    If $s=1$ then we can conclude that $|b_2 a_2| + |b_4 a_4| = |b_2 b_4| + |a_2 a_4|$ which implies that the newly constructed path is also optimal. 
    It follows via induction that any extra loop between $A$ and $B$ can be removed to construct an optimal hamiltonian path that visits all points in $B$ consecutively.
\end{proof}

\begin{corollary}\label{cor:wsp_tsp}
    Suppose we have two sets of points $A$ and $B$ that are $s$-well-separated where $s \geq 1$. Furthermore if $s > 1$ then all optimal TSP tours on $A \cup B$ must visit all points in $A$ before any point in $B$ (or vice versa).
\end{corollary}

\begin{rem*}
    The bound on $s$ in the previous corollary is tight.
\end{rem*}
\begin{proof}
    Fix any $s < 1$. 
    Next we define $p > -\frac{1}{\log_2(s)}$ and consider the lp norm on $\mathbb{R}^2$.
    We can then note that $s < 2^{-1/p}$.
    Now we can consider the following partition of the unit square $A=\{(0,0), (1,1)\}$ and $B = \{(0,1), (1,0)\}$.

    The diameter of each set is the length of the diagonal of the unit square within the lp norm. 
    This distance is equal to $(1^p + 1^p)^{1/p} = 2^{1/p}$.
    The distance between the two sets is one since the closest points between the two sets are one unit apart in either the x or y direction.
    Thus the separation factor of these two sets is $\frac{1}{2^{1/p}} = 2^{-1/p}$.
    Since $s < 2^{-1/p}$ we know that $s$ the two sets are $s$-well-separated.

    We can then note that the optimal TSP tour on $A \cup B$ has length $4$ and must be $((0,0) \to (0,1) \to (1,1) \to (1,0) \to (0,0))$ or $((0,0) \to (1,0) \to (1,1) \to (0,1) \to (0,0))$ both of which visit points in $A$ and $B$ alternately.
    Any tour which visits all points in $A$ before any point in $B$ (or vice versa) must have length at least $2 \cdot 2^{1/p} + 2$ which is strictly greater than the optimal tour which has length $4$.
\end{proof}

\subsubsection{Second Level Analysis}

\begin{lem}
    Suppose we have two sets of points $A$ and $B$ that are $s$-well-separated where $s \geq 1$. 
    Then the multiple optimal hamiltonian paths problem solution between starting nodes $a'_s,a''_s \in A$ and ending nodes $a'_e, a''_e \in A$ has a solution where one path only has points in $A$ and another visits all points in $B$ consecutively.
    Furthermore if $s > 1$ then all optimal solutions must satisfy this property.
\end{lem}
\begin{proof}
    Assume that for some solution to the problem the two optimal paths each visit some points in $B$.
    We will define the set of points in $A,B$ from the first path as $A^*, B^*$ and the set of points in $A,B$ from the second path as $A^\dagger, B^\dagger$.
    Then we can note that there must be the following edges in the first path:
    $(a^*_1, b^*_1), (b^*_2, a^*_2)$ and the following edges in the other path: 
    $(a^\dagger_1, b^\dagger_1)$ all ordered as they appear in their respective paths.
    WLOG let the first path have endpoints $a'_s$ and $a'_e$ and let the second path have endpoints $a''_s$ and $a''_e$.
    This gives the following partial structure of the paths:
    $$a'_s \leadsto a^*_1 \to b^*_1 \leadsto b^*_2 \to a^*_2 \leadsto a'_e
    \quad\quad
    a''_s \leadsto a^\dagger_1 \to b^\dagger_1 \leadsto b^\dagger_2 \to a^\dagger_2 \leadsto a''_e$$
    
    Since the paths are optimal we know that they satisfy the 2-optimality condition implying the following:
    $|a^*_1 b^*_1| + |b^\dagger_2 a^\dagger_2| \leq |a^*_1 a^\dagger_2| + |b^\dagger_2 b^*_1|$.
    This is to say that the following paths must have a combined length at least as long as the original paths:
    $$a'_s \leadsto a^*_1 \to a^\dagger_2 \leadsto a''_e
    \quad\quad
    a''_s \leadsto a^\dagger_1 \to b^\dagger_1 \leadsto b^\dagger_2 \to a^\dagger_2 \leadsto a''_e$$

    Since $A$ and $B$ are $s$-well-separated we know that $\forall i,j; |a_i a_j| \leq \frac{1}{s} |a_i b_i|$. From this we can conclude the following:
    \begin{align*}
        |a^*_1 b^*_1| + |b^\dagger_2 a^\dagger_2| &\leq |a^*_1 a^\dagger_2| + |b^\dagger_2 b^*_1| \\
        &\leq \frac{1}{s} |a^*_1 b^*_1| + \frac{1}{s} |b^\dagger_2 a^\dagger_2| \\
        &= \frac{1}{s} (|a^*_1 b^*_1| + |b^\dagger_2 a^\dagger_2|)
    \end{align*}

    This implies that $1 \leq \frac{1}{s}$ which in turn implies that $s \leq 1$. 
    If $s > 1$ this is a contradiction and therefore any solution where both paths visit some points in $B$ cannot be optimal.
    If $s=1$ then we can conclude that $|a^*_1 b^*_1| + |b^\dagger_2 a^\dagger_2| = |a^*_1 a^\dagger_2| + |b^\dagger_2 b^*_1|$ which implies that the newly constructed paths are also optimal.
    It follows via induction that all the extra loops between $A$ and $B$ can be removed from one of the paths to construct a solution where one path only has points in $A$ and another visits all points in $B$.

    From here we can examine the path that visits all the points in $B$ and take all the points it visits in $A$ and label this as $A'$.
    We can then note that $A'$ and $B$ are $s$-well-separated.
    By \Cref{thm:wsp_path_aa} we know that there exists an optimal hamiltonian path between the two endpoints in $A'$ that visits all points in $B$ consecutively and furthermore if $s > 1$ then all optimal hamiltonian paths must visit all points in $B$ consecutively.
\end{proof}

%\begin{corollary}
%    Suppose we have two sets of points $A$ and $B$ that are $s$-well-separated where $s \geq 1$. Then the multiple optimal hamiltonian paths problem solution between where all starting and ending nodes are in $A$ has a solution where all but one path has only points in $A$ and one path visits all points in $B$ consecutively. Furthermore if $s > 1$ then all optimal solutions must satisfy this property.
%\end{corollary}

\begin{lem}
    Suppose we have two sets of points $A$ and $B$ that are $s$-well-separated where $s \geq 1$. Then the multiple optimal hamiltonian paths problem solution between starting nodes $a'_s,a''_s \in A$ and ending nodes $a'_e \in A$ and $b'_e \in B$ has a solution where one path only has points in $A$ and another has exactly one edge connecting $A$ to $B$. Furthermore if $s > 1$ then all optimal solutions must satisfy this property.
\end{lem}
\begin{proof}
    Assume that for some solution to the problem the two optimal paths each visit some points in $B$.
    We will define the set of points in $A,B$ from the first path as $A^*, B^*$ and the set of points in $A,B$ from the second path as $A^\dagger, B^\dagger$.
    Then we can note that there must be the following edges in the first path:
    $(a^*_1, b^*_1), (b^*_2, a^*_2)$ and the following edges in the other path: 
    $(a^\dagger_1, b^\dagger_1)$ all ordered as they appear in their respective paths.
    WLOG let the first path have endpoints $a'_s$ and $a'_e$ and let the second path have endpoints $a''_s$ and $b'_e$.
    This gives the following partial structure of the paths:
    $$a'_s \leadsto a^*_1 \to b^*_1 \leadsto b^*_2 \to a^*_2 \leadsto a'_e
    \quad\quad
    a''_s \leadsto a^\dagger_1 \to b^\dagger_1 \leadsto b'_e$$

    Since the paths are optimal we know that they satisfy the 2-optimality condition implying the following:
    $|b^*_2 a^*_2| + |a^\dagger_1 b^\dagger_1| \leq |b^*_2 b^\dagger_1| + |a^\dagger_1 a^*_2|$.
    This is to say that the following paths must have a combined length at least as long as the original paths:
    $$a'_s \leadsto a^*_1 \to b^*_1 \leadsto b^*_2 \to b^\dagger_1 \leadsto b'_e
    \quad\quad
    a''_s \leadsto a^\dagger_1 \to a^*_2 \leadsto a'_e$$

    Since $A$ and $B$ are $s$-well-separated we know that $\forall i,j; |a_i a_j| \leq \frac{1}{s} |a_i b_i|$. From this we can conclude the following:
    \begin{align*}
        |b^*_2 a^*_2| + |a^\dagger_1 b^\dagger_1| &\leq |b^*_2 b^\dagger_1| + |a^\dagger_1 a^*_2| \\
        &\leq \frac{1}{s} |b^*_2 a^*_2| + \frac{1}{s} |a^\dagger_1 b^\dagger_1| \\
        &= \frac{1}{s} (|b^*_2 a^*_2| + |a^\dagger_1 b^\dagger_1|)
    \end{align*}

    This implies that $1 \leq \frac{1}{s}$ which in turn implies that $s \leq 1$.
    If $s > 1$ this is a contradiction and therefore any solution where both paths visit some points in $B$ cannot be optimal.
    If $s=1$ then we can conclude that $|b^*_2 a^*_2| + |a^\dagger_1 b^\dagger_1| = |b^*_2 b^\dagger_1| + |a^\dagger_1 a^*_2|$ which implies that the newly constructed paths are also optimal.
    It follows via induction that all the extra loops between $A$ and $B$ can be removed from one of the paths to construct a solution where one path only has points in $A$ and the other visits all points in $B$.

    From here we can examine the path that visits all the points in $B$ and take all the points it visits in $A$ and label this as $A'$.
    We can then note that $A'$ and $B$ are $s$-well-separated.
    By \Cref{thm:wsp_path_ab} we know that there exists an optimal hamiltonian path between the two endpoints in $A' \cup B$ that visits all points in $A'$ before any point in $B$ and furthermore if $s > 1$ then all optimal hamiltonian paths must visit all points in $A'$ before any point in $B$.
\end{proof}

\begin{lem}
    Suppose we have two sets of points $A$ and $B$ that are $s$-well-separated where $s \geq 1$. Then the multiple optimal hamiltonian paths problem solution between starting nodes $a'_s,a''_s \in A$ and ending nodes $b'_e,b''_e \in B$ has a solution where each path has exactly one edge connecting $A$ to $B$. Furthermore if $s > 1$ then all optimal solutions must satisfy this property.
\end{lem}
\begin{proof}
    Suppose that for some solution one of its paths has more than one edge connecting $A$ to $B$.
    Then we can define the set of points in $A,B$ from the first path as $A^*, B^*$. 
    Then we can note that $A^*$ and $B^*$ are still $s$-well-separated since they are subsets of $A$ and $B$ respectively.
    By \Cref{thm:wsp_path_ab} we know that there exists an optimal hamiltonian path between the two endpoints in $A^* \cup B^*$ that only has a single edge connecting $A^*$ to $B^*$ and furthermore if $s > 1$ then all optimal hamiltonian paths must only have a single edge connecting $A^*$ to $B^*$.
    If $s=1$ and the other path also has more than one edge connecting $A$ to $B$ then we can apply the same argument to construct an optimal solution where both paths only have a single edge connecting $A$ to $B$.
\end{proof}


\subsubsection{Full Structure Analysis}

\begin{thm}\label{thm:bin_wspd_tour_structure}
    Let $P$ be a set of points and suppose we have a WSPD $\PairDecomp_s$ of $P$ with separation factor $s > 1$. 
    Then for any given $k$-binary WSP $(A_i, B_i) \in \PairDecomp_s$ there must exist exactly one or two edges in the optimal TSP tour on $P$ connecting them. Furthermore, there are exactly two edges exiting $A_i \cup B_i$ if they are not a $1$-binary WSP (aka the root pair).
\end{thm}
\begin{proof}
    This can be proved via induction. 
    For the top level pair of $(A_t, B_t) \in \PairDecomp_s$ where $A_t \cup B_t = P$ the result follows directly from Corollary~\ref{cor:wsp_tsp} which implies exactly two edges connect $A_t$ to $B_t$.
    
    Next consider the second level cluster $(A_t', B_t') \in \PairDecomp_s$. WLOG let $A_t' \cup B_t' = A_t$.
    By the previous case we know that there are exactly two edges exiting $A_t$. WLOG let one of these edges be of the form $(a_*, d_*)$ where $a_* \in A_t'$ and $d_* \in B_t$. We proceed with a case analysis.

    \textbf{Case 1:} The second edge exiting $A_t$ is of the form $(b_*, d_\dagger)$ where $b_* \in B_t'$ and $d_\dagger \in B_t$.
    Since there are no other edges exiting $A_t$ we know that the optimal TSP tour must contain a hamiltonian path on $A_t$ with endpoints $a_*$ and $b_*$.
    By Theorem~\ref{thm:wsp_path_ab} we know that this path must have exactly one edge connecting $A_t'$ to $B_t'$.

    \textbf{Case 2:} The second edge exiting $A_t$ is of the form $(a_\dagger, d_\dagger)$ where $a_\dagger \in A_t'$ and $d_\dagger \in B_t$.
    Since there are no other edges exiting $A_t$ we know that the optimal TSP tour must contain a hamiltonian path on $A_t$ with endpoints $a_*$ and $a_\dagger$.
    By Theorem~\ref{thm:wsp_path_aa} we know that this path must have exactly two edges connecting $A_t'$ to $B_t'$.

    Now consider the third level or lower pair $(A_i, B_i) \in \PairDecomp_s$. Assume by induction that for the parent pair $(C_p, D_p) \in \PairDecomp_s$ with $A_i \cup B_i = C_p$ there is exactly one or two edges connecting $C_p$ to $D_p$ in the optimal TSP tour and furthermore there are exactly two edges exiting $C_p$.

    Let $(c_*, d_*)$ be an edge connecting $C_p$ to $D_p$ in the optimal TSP tour.
    WLOG assume that $c_* \in A_i$ and define $a_* = c_*$ (the case where $c_* \in B_i$ is symmetric).
    We proceed with a case analysis.

    \textbf{Case 1:} There are two edges connecting $C_p$ to $D_p$ in the optimal TSP tour and the second edge is of the form $(b_*, d_\dagger)$ where $b_* \in B_i$ and $d_\dagger \in D_p$.
    Since there are no other edges exiting $C_p$ we know that the optimal TSP tour must contain a hamiltonian path on $C_p$ with endpoints $a_*$ and $b_*$.
    By Theorem~\ref{thm:wsp_path_ab} we know that this path must visit all points in $A_i$ before any point in $B_i$ (or vice versa).
    Thus there must be exactly one edge connecting $A_i$ to $B_i$. And exactly two edges $(a_*, d_*)$, $(b_*, d_\dagger)$ exiting $C_p$ as desired.

    \textbf{Case 2:} There are two edges connecting $C_p$ to $D_p$ in the optimal TSP tour and the second edge is of the form $(a_\dagger, d_\dagger)$ where $a_\dagger \in A_i$ and $d_\dagger \in D_p$.
    Since there are no other edges exiting $C_p$ we know that the optimal TSP tour must contain a hamiltonian path on $C_p$ with endpoints $a_*$ and $a_\dagger$.
    By Theorem~\ref{thm:wsp_path_aa} we know that this path must contain exactly two edges between $A_i$ and $B_i$. Furthermore there are exactly two edges $(a_*, d_*)$, $(a_\dagger, d_\dagger)$ exiting $C_p$ as desired.
    \textbf{Case 3:} There is only one edge connecting $C_p$ to $D_p$ in the optimal TSP tour and the other edge is of the form $(a_\dagger, e_*)$ where $a_\dagger \in A_i$ and $e_* \notin C_p \cup D_p$.
    Since there are no other edges exiting $C_p$ we know that the optimal TSP tour must contain a hamiltonian path on $C_p$ with endpoints $a_*$ and $a_\dagger$.
    By Theorem~\ref{thm:wsp_path_aa} we know that this path must only contain two edges between $A_i$ and $B_i$. Furthermore there are exactly two edges $(a_*, d_*)$, $(a_\dagger, e_*)$ exiting $C_p$ as desired.

    \textbf{Case 4:} There is only one edge connecting $C_p$ to $D_p$ in the optimal TSP tour and the other edge is of the form $(b_*, e_*)$ where $b_* \in B_i$ and $e_* \notin C_p \cup D_p$.
    Since there are no other edges exiting $C_p$ we know that the optimal TSP tour must contain a hamiltonian path on $C_p$ with endpoints $a_*$ and $b_*$.
    By Theorem~\ref{thm:wsp_path_ab} we know that this path must visit all points in $A_i$ before any point in $B_i$ (or vice versa).
    Thus there must be exactly one edges connecting $A_i$ to $B_i$. And exactly two edges $(a_*, d_*)$, $(b_*, e_*)$ exiting $C_p$ as desired.

    Therefore by induction we have shown that for any given pair $(A_i, B_i) \in \PairDecomp_s$ there must exist exactly one or two edges in the optimal TSP tour on $P$ connecting them. Furthermore, there are exactly two edges exiting $A_i \cup B_i$ if they are not the root clusters.
\end{proof}

There may be a way to deterministically say which of the two cases will occur based on the structure of the WSPD but I have not yet found it.

\subsection{Moving beyond binary WSPDs}

\emph{TODO: Update these to have $s=1.0$}

\begin{thm}
    Let $P = A \cup B \cup C$ be a set of points where $A,B,C$ are pairwise $s$-well-separated with $s > 1$. 
    Then a hamiltonian path which starts on a point $a' \in A$ and ends on a point $a'' \in A$ must have only three or four edges connecting the three sets.
\end{thm}
\begin{proof}
    To prove this we can observe two different cases. 

    \textbf{Case 1:} The optimal hamiltonian path on $P$ contains at least one edge connecting each pair of sets.
    WLOG let these edges be $(a_1, b_1), (b_2, c_1), (c_2, a_2)$ ordered as they appear in the path (if $c$ occurs before $b$ then flip the names).
    Assume by contradiction that the optimal hamiltonian path from $a'$ to $a''$ over $P$ requires more than four edges connecting the three sets.
    Then there must either be another loop connecting the three sets or there must be at least two more edges connecting one of the pairs of sets.

    \textbf{Subcase 1a:} There is another loop connecting the three sets both in the same orientation.
    Then the optimal path must contain at least six edges connecting the three sets.
    WLOG let these edges be $(a_1, b_1)$, $(b_2, c_1)$, $(c_2, a_2)$, $(a_3, b_3)^*$, $(b_4, c_3)^*$, $(c_4, a_4)^*$ ordered as they appear in the path. This gives the following partial structure of the path:
    $$a' \to a_1 \to b_1 \leadsto b_2 \to c_1 \leadsto c_2 \to a_2 \leadsto a_3 \to b_3 \leadsto b_4 \to c_3 \leadsto c_4 \to a_4 \leadsto a''$$

    Since the path is optimal we know that it satisfies the 6-optimality condition implying the following:
    $|a_1 b_1| + |b_2 c_1| + |c_2 a_2| + |a_3 b_3| + |b_4 c_3| + |c_4 a_4| \leq |a_1 a_2| + |a_3 b_3| + |b_4 b_1| + |b_2 c_1| + |c_2 c_3| + |c_4 a_4|$.
    We can then rearrange this to get:
    \begin{align*}
        |a_1 b_1| + |c_2 a_2| + |b_4 c_3| &\leq |a_1 a_2| + |b_4 b_1| + |c_2 c_3| \\
        &\leq \frac{1}{s} |a_1 b_1| + \frac{1}{s} |c_2 a_2| + \frac{1}{s} |b_4 c_3| \\
        &= \frac{1}{s} (|a_1 b_1| + |c_2 a_2| + |b_4 c_3|)
    \end{align*}
    This implies that $1 \leq \frac{1}{s}$ which in turn implies that $s \leq 1$ contradicting our original statement that $s > 1$.

    \textbf{Subcase 1b:} There is another loop connecting the three sets in the opposite orientation.
    Then the optimal path must contain at least six edges connecting the three sets.
    WLOG let these edges be $(a_1, b_1)$, $(b_2, c_1)$, $(c_2, a_2)$, $(a_3, c_3)^*$, $(c_4, b_3)^*$, $(b_4, a_4)^*$ ordered as they appear in the path. This gives the following partial structure of the path:
    $$a' \to a_1 \to b_1 \leadsto b_2 \to c_1 \leadsto c_2 \to a_2 \leadsto a_3 \to c_3 \leadsto c_4 \to b_3 \leadsto b_4 \to a_4 \leadsto a''$$

    Since the path is optimal we know that it satisfies the 6-optimality condition implying the following:
    $|a_1 b_1| + |b_2 c_1| + |c_2 a_2| + |a_3 c_3| + |c_4 b_3| + |b_4 a_4| \leq |a_1 a_2| + |a_3 c_3| + |c_4 c_2| + |c_1 b_2| + |b_1 b_3| + |b_4 a_4|$.
    We can then rearrange this to get:
    \begin{align*}
        |a_1 b_1| + |c_2 a_2| + |c_4 b_3| &\leq |a_1 a_2| + |c_2 c_4| + |b_1 b_3| \\
        &\leq \frac{1}{s} |a_1 b_1| + \frac{1}{s} |c_2 a_2| + \frac{1}{s} |c_4 b_3| \\
        &= \frac{1}{s} (|a_1 b_1| + |c_2 a_2| + |c_4 b_3|)
    \end{align*}
    This implies that $1 \leq \frac{1}{s}$ which in turn implies that $s \leq 1$ contradicting our original statement that $s > 1$.

    \textbf{Subcase 1c:} There must be at least two additional edges at the end causing a loop from $A$ to $B$.
    Then the optimal hamiltonian path must contain at least five edges connecting the three sets.
    WLOG let these edges be $(a_1, b_1)$, $(b_2, c_1)$, $(c_2, a_2)$, $(a_3, b_3)^*$, $(b_4, a_4)^*$ ordered as they appear in the path. 
    This gives the following partial structure of the path:
    $$a' \to a_1 \to b_1 \leadsto b_2 \to c_1 \leadsto c_2 \to a_2 \leadsto a_3 \to b_3 \leadsto b_4 \to a_4 \leadsto a''$$

    Since the path is optimal we know that it satisfies the 2-optimality condition implying the following:
    \begin{align*}
        |a_1 b_1| + |a_3 b_3| &\leq |a_1 a_3| + |b_1 b_3| \\
        &\leq \frac{1}{s} |a_1 b_1| + \frac{1}{s} |a_3 b_3| \\
        &= \frac{1}{s} (|a_1 b_1| + |a_3 b_3|)
    \end{align*}
    This implies that $1 \leq \frac{1}{s}$ which in turn implies that $s \leq 1$ contradicting our original statement that $s > 1$.

    \textbf{Subcase 1d:} There must be at least two additional edges at the end causing a loop from $A$ to $C$.
    Then the optimal hamiltonian path must contain at least five edges connecting the three sets.
    WLOG let these edges be $(a_1, b_1)$, $(b_2, c_1)$, $(c_2, a_2)$, $(a_3, c_3)^*$, $(c_4, a_4)^*$ ordered as they appear in the path.
    This gives the following partial structure of the path:
    $$a' \to a_1 \to b_1 \leadsto b_2 \to c_1 \leadsto c_2 \to a_2 \leadsto a_3 \to c_3 \leadsto c_4 \to a_4 \leadsto a''$$

    Since the path is optimal we know that it satisfies the 2-optimality condition implying the following:
    \begin{align*}
        |c_2 a_2| + |c_4 a_4| &\leq |c_2 c_4| + |a_2 a_4| \\
        &\leq \frac{1}{s} |c_2 a_2| + \frac{1}{s} |c_4 a_4| \\
        &= \frac{1}{s} (|c_2 a_2| + |c_4 a_4|)
    \end{align*}
    This implies that $1 \leq \frac{1}{s}$ which in turn implies that $s \leq 1$ contradicting our original statement that $s > 1$.


    \textbf{Subcase 1e:} There must be at least two additional edges connecting $A,B$ before starting the primary loop.
    Then the optimal hamiltonian path must contain at least five edges connecting the three sets.
    WLOG let these edges be $(a_3, b_3)^*$, $(b_4, a_4)^*$, $(a_1, b_1)$, $(b_2, c_1)$, $(c_2, a_2)$ ordered as they appear in the path.
    This gives the following partial structure of the path:
    $$a' \to a_3 \to b_3 \leadsto b_4 \to a_4 \leadsto a_1 \to b_1 \leadsto b_2 \to c_1 \leadsto c_2 \to a_2 \leadsto a''$$

    Since the path is optimal we know that it satisfies the 2-optimality condition implying the following:
    \begin{align*}
        |a_3 b_3| + |a_1 b_1| &\leq |a_3 a_1| + |b_3 b_1| \\
        &\leq \frac{1}{s} |a_3 b_3| + \frac{1}{s} |a_1 b_1| \\
        &= \frac{1}{s} (|a_3 b_3| + |a_1 b_1|)
    \end{align*}
    This implies that $1 \leq \frac{1}{s}$ which in turn implies that $s \leq 1$ contradicting our original statement that $s > 1$.

    \textbf{Subcase 1f:} There must be at least two additional edges connecting $A,C$ before starting the primary loop.
    Then the optimal hamiltonian path must contain at least five edges connecting the three sets.
    WLOG let these edges be $(a_3, c_3)^*$, $(c_4, a_4)^*$, $(a_1, b_1)$, $(b_2, c_1)$, $(c_2, a_2)$ ordered as they appear in the path.
    This gives the following partial structure of the path:
    $$a' \to a_3 \to c_3 \leadsto c_4 \to a_4 \leadsto a_1 \to b_1 \leadsto b_2 \to c_1 \leadsto c_2 \to a_2 \leadsto a''$$

    Since the path is optimal we know that it satisfies the 2-optimality condition implying the following:
    \begin{align*}
        |c_4 a_4| + |c_2 a_2| &\leq |c_4 c_2| + |a_4 a_2| \\
        &\leq \frac{1}{s} |c_4 a_4| + \frac{1}{s} |c_2 a_2| \\
        &= \frac{1}{s} (|c_4 a_4| + |c_2 a_2|)
    \end{align*}
    This implies that $1 \leq \frac{1}{s}$ which in turn implies that $s \leq 1$ contradicting our original statement that $s > 1$.

    \textbf{Subcase 1g:} There is a loop back from $B$ to $C$ before the primary loop is ended.
    Then the optimal hamiltonian path must contain at least five edges connecting the three sets.
    WLOG let these edges be $(a_1, b_1)$, $(b_2, c_1)$, $(c_4, b_3)^*$, $(b_4, c_3)^*$, $(c_2, a_2)$ ordered as they appear in the path.
    This gives the following partial structure of the path:
    $$a' \to a_1 \to b_1 \leadsto b_2 \to c_1 \leadsto c_4 \to b_3 \leadsto b_4 \to c_3 \leadsto c_2 \to a_2 \leadsto a''$$

    Since the path is optimal we know that it satisfies the 2-optimality condition implying the following:
    \begin{align*}
        |b_2 c_1| + |b_4 c_3| &\leq |b_2 b_4| + |c_1 c_3| \\
        &\leq \frac{1}{s} |b_2 c_1| + \frac{1}{s} |b_4 c_3| \\
        &= \frac{1}{s} (|b_2 c_1| + |b_4 c_3|)
    \end{align*}
    This implies that $1 \leq \frac{1}{s}$ which in turn implies that $s \leq 1$ contradicting our original statement that $s > 1$.

    \textbf{Case 2:} The optimal TSP tour on $P$ does not contain an edge connecting each pair of sets.
    \textbf{Subcase 2a:} WLOG let there only edges connecting $A$ to $B$ and $B$ to $C$ ($C$ and $B$ can be alternated symmetrically).
    Then the optimal hamiltonian path must contain at least four edges connecting the three sets.
    WLOG let these edges be $(a_1, b_1)$, $(b_2, c_1)$, $(c_2, b_3)$, $(b_4, a_2)$ ordered as they appear in the path.

    \textbf{Subsubcase 2a.i:} There is an extra loop connecting $A$ to $B$ before ending the path.
    Then the optimal hamiltonian path must contain at least six edges connecting the three sets.
    WLOG let these edges be $(a_1, b_1)$, $(b_5, a_3)^*$, $(a_4, b_6)^*$, $(b_2, c_1)$, $(c_2, b_3)$, $(b_4, a_2)$ ordered as they appear in the path.
    This gives the following partial structure of the path:
    $$a' \leadsto a_1 \to b_1 \leadsto b_5 \to a_3 \leadsto a_4 \to b_6 \leadsto b_2 \to c_1 \leadsto c_2 \to b_3 \leadsto b_4 \to a_2 \leadsto a''$$

    Since the path is optimal we know that it satisfies the 2-optimality condition implying the following:
    \begin{align*}
        |a_1 b_1| + |a_4 b_6| &\leq |a_1 a_4| + |b_1 b_6| \\
        &\leq \frac{1}{s} |a_1 b_1| + \frac{1}{s} |a_4 b_6| \\
        &= \frac{1}{s} (|a_1 b_1| + |a_4 b_6|)
    \end{align*}
    This implies that $1 \leq \frac{1}{s}$ which in turn implies that $s \leq 1$ contradicting our original statement that $s > 1$.
    
    \textbf{Subsubcase 2a.ii:} There is an extra loop connecting $B$ to $C$ before ending the path.
    Then the optimal hamiltonian path must contain at least six edges connecting the three sets.
    WLOG let these edges be $(a_1, b_1)$, $(b_2, c_1)$, $(c_3, b_5)^*$, $(b_6, c_4)^*$, $(c_2, b_3)$, $(b_4, a_2)$ ordered as they appear in the path.
    This gives the following partial structure of the path:
    $$a' \leadsto a_1 \to b_1 \leadsto b_2 \to c_1 \leadsto c_3 \to b_5 \leadsto b_6 \to c_4 \leadsto c_2 \to b_3 \leadsto b_4 \to a_2 \leadsto a''$$

    Since the path is optimal we know that it satisfies the 2-optimality condition implying the following:
    \begin{align*}
        |b_2 c_1| + |b_6 c_4| &\leq |b_2 b_6| + |c_1 c_4| \\
        &\leq \frac{1}{s} |b_2 c_1| + \frac{1}{s} |b_6 c_4| \\
        &= \frac{1}{s} (|b_2 c_1| + |b_6 c_4|)
    \end{align*}
    This implies that $1 \leq \frac{1}{s}$ which in turn implies that $s \leq 1$ contradicting our original statement that $s > 1$.

    \textbf{Subsubcase 2a.iii:} There is an extra loop connecting $A,B$ after ending the path.
    Then the optimal hamiltonian path must contain at least six edges connecting the three sets.
    WLOG let these edges be $(a_1, b_1)$, $(b_2, c_1)$, $(c_2, b_3)$, $(b_4, a_2)$, $(a_3, b_5)^*$, $(b_6, a_4)^*$ ordered as they appear in the path.
    This gives the following partial structure of the path:
    $$a' \leadsto a_1 \to b_1 \leadsto b_2 \to c_1 \leadsto c_2 \to b_3 \leadsto b_4 \to a_2 \leadsto a_3 \to b_5 \leadsto b_6 \to a_4 \leadsto a''$$

    Since the path is optimal we know that it satisfies the 2-optimality condition implying the following:
    \begin{align*}
        |a_1 b_1| + |a_3 b_5| &\leq |a_1 a_3| + |b_1 b_5| \\
        &\leq \frac{1}{s} |a_1 b_1| + \frac{1}{s} |a_3 b_5| \\
        &= \frac{1}{s} (|a_1 b_1| + |a_3 b_5|)
    \end{align*}
    This implies that $1 \leq \frac{1}{s}$ which in turn implies that $s \leq 1$ contradicting our original statement that $s > 1$.

    \textbf{Subcase 2b:} Let there only be edges connecting $A$ to $B$ and $A$ to $C$.
    WLOG let these edges be $(a_1, b_1)$, $(b_2, a_2)$, $(a_3, c_1)$, $(c_2, a_4)$ ordered as they appear in the path (if $C$ occurs before $B$ then flip the names).

    \textbf{Subsubcase 2b.i:} There is an extra loop connecting $A$ to $B$ before ending the path.
    Then the optimal hamiltonian path must contain at least six edges connecting the three sets.
    WLOG let these edges be $(a_1, b_1)$, $(b_5, a_3)^*$, $(a_4, b_6)^*$, $(b_2, a_2)$, $(a_5, c_1)$, $(c_2, a_6)$ ordered as they appear in the path.
    This gives the following partial structure of the path:
    $$a' \leadsto a_1 \to b_1 \leadsto b_5 \to a_3 \leadsto a_4 \to b_6 \leadsto b_2 \to a_2 \leadsto a_5 \to c_1 \leadsto c_2 \to a_6 \leadsto a''$$

    Since the path is optimal we know that it satisfies the 2-optimality condition implying the following:
    \begin{align*}
        |a_1 b_1| + |a_4 b_6| &\leq |a_1 a_4| + |b_1 b_6| \\
        &\leq \frac{1}{s} |a_1 b_1| + \frac{1}{s} |a_4 b_6| \\
        &= \frac{1}{s} (|a_1 b_1| + |a_4 b_6|)
    \end{align*}
    This implies that $1 \leq \frac{1}{s}$ which in turn implies that $s \leq 1$ contradicting our original statement that $s > 1$.

    \textbf{Subsubcase 2b.ii:} There is an extra loop connecting $A,C$ after ending the path.
    Then the optimal hamiltonian path must contain at least six edges connecting the three sets.
    WLOG let these edges be $(a_1, b_1)$, $(b_2, a_2)$, $(a_3, c_1)$, $(c_2, a_4)$, $(a_5, c_3)^*$, $(c_4, a_6)^*$ ordered as they appear in the path.
    This gives the following partial structure of the path:
    $$a' \leadsto a_1 \to b_1 \leadsto b_2 \to a_2 \leadsto a_3 \to c_1 \leadsto c_2 \to a_4 \leadsto a_5 \to c_3 \leadsto c_4 \to a_6 \leadsto a''$$

    Since the path is optimal we know that it satisfies the 2-optimality condition implying the following:
    \begin{align*}
        |a_3 c_1| + |a_5 c_3| &\leq |a_3 a_5| + |c_1 c_3| \\
        &\leq \frac{1}{s} |a_3 c_1| + \frac{1}{s} |a_5 c_3| \\
        &= \frac{1}{s} (|a_3 c_1| + |a_5 c_3|)
    \end{align*}
    This implies that $1 \leq \frac{1}{s}$ which in turn implies that $s \leq 1$ contradicting our original statement that $s > 1$.

    \textbf{Subsubcase 2b.iii:} There is an extra loop connecting $A,B$ before ending the path.
    Then the optimal hamiltonian path must contain at least six edges connecting the three sets.
    WLOG let these edges be $(a_1, b_1)$, $(b_2, a_2)$, $(a_3, c_1)$, $(c_2, a_4)$, $(a_5, b_5)^*$, $(b_6, a_6)^*$ ordered as they appear in the path.
    This gives the following partial structure of the path:
    $$a' \leadsto a_1 \to b_1 \leadsto b_2 \to a_2 \leadsto a_3 \to c_1 \leadsto c_2 \to a_4 \leadsto a_5 \to b_5 \leadsto b_6 \to a_6 \leadsto a''$$

    Since the path is optimal we know that it satisfies the 2-optimality condition implying the following:
    \begin{align*}
        |a_1 b_1| + |a_5 b_5| &\leq |a_1 a_5| + |b_1 b_5| \\
        &\leq \frac{1}{s} |a_1 b_1| + \frac{1}{s} |a_5 b_5| \\
        &= \frac{1}{s} (|a_1 b_1| + |a_5 b_5|)
    \end{align*}
    This implies that $1 \leq \frac{1}{s}$ which in turn implies that $s \leq 1$ contradicting our original statement that $s > 1$.

    Therefore in all cases we have reached a contradiction when assuming that there are more than four edges connecting the three sets.
    Thus the optimal hamiltonian path from $a'$ to $a''$ over $P$ must have only three or four edges connecting the three sets.
\end{proof}

\begin{corollary}
    Let $P = A \cup B \cup C$ be a set of points where $A,B,C$ are pairwise $s$-well-separated with $s > 1$. 
    Then $A$, $B$, and $C$ must be TSP-separated.
\end{corollary}

\section{Algorithms}

\begin{algorithm}[H]
\caption{\textsc{BestBalancedBiPartition}$(D,s)$}
\label{alg:balanced-bipartition}
\begin{algorithmic}[1]
\Require A metric distance matrix $D \in \mathbb{R}^{n \times n}$ (symmetric, triangle inequality), separation factor $s > 1$
\Ensure A bi-partition $(A,B)$ that is $s$-well-separated and maximally balanced, or \textsc{NULL} if no such partition exists

\State $E \gets \textsc{BuildMST}(D)$ \Comment{$n-1$ edges}
\State $\text{bestScore} \gets +\infty$
\State $\text{bestPartition} \gets \textsc{NULL}$

\ForAll{edges $e = (u,v)$ in $E$}
    \State $(A,B) \gets \textsc{ConnectedComponents}(E \setminus \{e\})$
    \State $\text{balance} \gets \max(|A|, |B|)$ \Comment{Balance score: size of larger cluster}
    \If{$\text{balance} \geq \text{bestScore}$}
        \State \textbf{continue} \Comment{Skip if not better than current best}
    \EndIf

    \State $\delta \gets w(e)$ \Comment{Single-linkage separation distance}

    \State $\operatorname{diam}(A) \gets \textsc{ComputeDiameter}(A,D)$
    \State $\operatorname{diam}(B) \gets \textsc{ComputeDiameter}(B,D)$
    \State $M \gets \max\{\operatorname{diam}(A), \operatorname{diam}(B)\}$

    \If{$\delta \ge s \cdot M$} \Comment{$s$-well-separated condition}
        \State $\text{bestScore} \gets \text{balance}$
        \State $\text{bestPartition} \gets (A,B)$
        \If{$\text{bestScore} = \lceil n/2 \rceil$} \Comment{Perfectly balanced partition found}
            \State \textbf{break}
        \EndIf
    \EndIf
\EndFor

\State \Return $\text{bestPartition}$
\end{algorithmic}
\end{algorithm}

Analyzing the program there are $n-1$ edges in the MST which takes $O(n^2)$ time to compute, and for each edge we compute the connected components in $O(n)$ time, compute the diameter of each component in $O(n^2)$ time, and check the separation condition in $O(1)$ time. Thus the overall runtime is $O(n^3)$.

\begin{lem}\label{lem:mst-contains-min-edge}
    Let $(A,B)$ be a bi-partition of a set of points $P$ within a metric space. Then any MST of $P$ must contain at least on edge connecting the two clusters $A$ and $B$ that is minimal among all edges connecting $A$ and $B$.
\end{lem}
\begin{proof}
    Let $e = (u,v)$ be an edge representing the minimum distance between $A$ and $B$ where $u \in A$ and $v \in B$.
    Suppose that $e$ is not in some MST of $P$. 
    Then by adding $e$ to the MST we create a cycle.
    Since $u \in A$ and $v \in B$ there must exist another edge $e' = (u', v')$ in the cycle where $u' \in A$ and $v' \in B$.
    Since $e$ represents the minimum distance between $A$ and $B$ we have that $\omega(e) \leq \omega(e')$.
    However by the minimax property of the MST we know that $\omega(e') \leq \omega(e)$.
    Thus $\omega(e) = \omega(e')$ and therefore the MST contains an edge representing the minimum distance between $A$ and $B$.
\end{proof}

\begin{lem}\label{lem:multiple-cuts-bound-separation}
    Let $(A,B)$ be a bi-partition of a set of points $P$ within a metric space. Suppose that more than one cut of any MST is required to separate $A$ and $B$. Then $(A,B)$ cannot be $s$-well-separated for any $s > 1$.
\end{lem}
\begin{proof}
    Let $e = (u, v)$ be the edge in the MST representing the minimum distance between $A$ and $B$ where $u \in A$ and $v \in B$ which must exist by \Cref{lem:mst-contains-min-edge}.
    Since more than one cut is required to separate $A$ and $B$, there must be another edge $e' = (v', u')$ in the MST where $u' \in A$ and $v' \in B$.
    Let $M = \max\{\diam(A), \diam(B)\}$.
    Define $e^* = (u, u')$ then we know that $\omega(e^*) \leq M$ by the definition of diameter.
    However by the minimax property of the MST we have that $\omega(e) \leq \omega(e^*)$.
    Thus $\omega(e) \leq M$ and therefore $(A,B)$ cannot be $s$-well-separated for any $s > 1$.
\end{proof}

\begin{thm}
    If there exists a bi-partition of the points that is $s$-well-separated (where $s>1$), then \Cref{alg:balanced-bipartition} will find a bi-partition that is $s$-well-separated and maximally balanced.
\end{thm}
\begin{proof}
    It follows from \Cref{lem:multiple-cuts-bound-separation} that since $s>1$ then any bi-partition that is $s$-well-separated must be separable by a single cut of any MST.
    Since \Cref{alg:balanced-bipartition} iterates through all edges of the MST and checks the bi-partition induced by cutting that edge, it will find all bi-partitions that are $s$-well-separated.
    Among these bi-partitions, the algorithm keeps track of the one with the best balance score (size of larger cluster) and returns it at the end.
    Therefore if there exists a bi-partition that is $s$-well-separated, the algorithm will find and return a bi-partition that is $s$-well-separated and maximally balanced.
\end{proof}

\begin{thm}
    Any algorithm that finds the best balanced bi-partition that is $s$-well-separated for $s\leq1$ is NP-hard.
\end{thm}
\begin{proof}
    This can be shown by a reduction from the Partition problem.
    Given a multiset of $n$ positive integers $S = \{x_1, x_2, \dots, x_n\}$, the Partition problem asks whether $S$ can be
    partitioned into two subsets $S_1$ and $S_2$ such that the sum of the elements in each subset is equal.
    We can construct a special set of points $P$ in the following way: for each integer $x_i$ in $S$, create $x_i$ points that are clumped on a unique vertex of the $n$ simplex. 
    Assume that the solution to the partition problem has each partition made up of at least two integers (if not we can trivially check for this case).
    Since every clump lies on a unique vertex of the simplex, the diameter of any partition is 1 and the distance between any two partitions is 1 thus making the separation factor exactly 1.
    Therefore any bi-partition of $P$ is 1-well-separated.
    It can then be noted that after solving for the best balanced bi-partition of $P$ that is $s$-well-separated for $s \leq 1$, if the resulting bi-partition has both clusters with equal size, then the answer to the Partition problem is yes and if not then the answer must be no.
\end{proof}

\subsubsection{Second Version}

\begin{algorithm}[H]
\caption{\textsc{BalancedMetricSplit2}$(D,s,k)$}
\label{alg:balanced_metric_split2}
\begin{algorithmic}[1]
\Require A metric distance matrix $D \in \mathbb{R}^{n\times n}$ (symmetric, triangle inequality), separation factor $s>0$, and number of clusters $k \ge 2$
\Ensure A no more than $k$-partition $\mathcal{P}=\{S_1,\dots,S_k\}$ that maximizes balance such that every pair of clusters is $s$-well-separated (or a single cluster of all points is returned)

\State $\mathcal{M} \gets \{D_{ij} \mid 1 \le i < j \le n\}$ \Comment{candidate diameter thresholds}
\State $\mathcal{S}^\star \gets \big\{ \{1,2,\dots,n\} \big\}$; $\text{bestScore} \gets n$

\ForAll{$M \in \mathcal{M}$}
    \State $E_M \gets \{(i,j)\mid 1\le i<j\le n,\; D_{ij} < s\cdot M\}$
    \State $G_M \gets (V=\{1,\dots,n\}, E_M)$
    \State $\mathcal{C} \gets \textsc{ConnectedComponents}(G_M)$

    \If{$1 < |\mathcal{C}| \leq l$}
        \State \textbf{continue}
    \EndIf

    \Comment{Diameter feasibility: ensure each component has diameter $\le M$}
    \State $\text{ok} \gets \textbf{true}$
    \ForAll{$C \in \mathcal{C}$}
        \If{$\diam(C) > M$}
            \State $\text{ok} \gets \textbf{false}$
            \State \textbf{break}
        \EndIf
    \EndFor
    \If{\textbf{not} ok}
        \State \textbf{continue}
    \EndIf

    \Comment{Separation feasibility: ensure pairwise separation}
    \State $\text{sep} \gets \textbf{true}$
    \ForAll{distinct $C, C' \in \mathcal{C}$}
        \State $\Delta(C,C') \gets \min\{D_{xy}\mid x\in C,\; y\in C'\}$
        \If{$\Delta(C,C') < s \cdot \max(\diam(C),\diam(C'))$}
            \State $\text{sep} \gets \textbf{false}$
            \State \textbf{break}
        \EndIf
    \EndFor
    \If{\textbf{not} sep}
        \State \textbf{continue}
    \EndIf

    \Comment{Balance objective: minimize size of largest cluster}
    \State $\text{score} \gets \max_{C\in\mathcal{C}} \bigl|C\bigr|$
    \If{$\text{score} < \text{bestScore}$}
        \State $\text{bestScore} \gets \text{score}$
        \State $\mathcal{S}^\star \gets \mathcal{C}$
    \EndIf
\EndFor

\State \Return $\mathcal{S}^\star$
\end{algorithmic}
\end{algorithm}

\appendix
\section{Outdated proofs}
% this section is for if a better bound is found (ie proof is still valid but not as strong as desired)
\begin{thm}[There is a better bound]
    Suppose we have two sets of points $A$ and $B$ that are $s$-well-separated where $s > 2.5$. Then $A$ and $B$ must be TSP-separated.
\end{thm}

\begin{rem*}
    The previous corollary cannot be guaranteed for $s \leq 0.5$
\end{rem*}
\begin{proof}
    If $s \leq 0.5$ a evil problem can be created in 1d. 
    Consider $A=\{0,1\}$ and $B=\{0.5,1.5\}$. 
    The diameter of each set is 1 and distance between the two sets is 0.5 making them 0.5-well-separated.
    However the optimal TSP tour on $A \cup B$ is $0 \to 0.5 \to 1 \to 1.5 \to 0$ which would contradict \Cref{cor:wsp_tsp} if it applied for $s \leq 0.5$.
\end{proof}

\section{Dead ends}
% these theorems are false

%Let $A$ and $B$ be two finite sets of points that are TSP-separated. 
%Let $\Cover_k(S)$ denote the optimal cost of $k$ disjoint hamiltonian paths covering all points in set $S$.
%Let $\BigLength_{2k}(A,B)$ be the sum of the $2k$ shortest edges between $A$ and $B$ without point replacement.
%Then, $\Cover_1(A) + \Cover_1(B) + \BigLength_{2}(A,B) \leq \Cover_k(A) + \Cover_k(B) + \BigLength_{2k}(A,B)$.

%Let $A$ and $B$ be two finite sets of points that are TSP-separated. Then $\dist(A,B) \geq \max(\lambda(A), \lambda(B))$.
%[Where $\lambda(A) = \max(|a_1 a_2|, |a_2 a_3|, \dots, |a_{n-1} a_n|)$ so that it represents the longest edge in the hamiltonian path through $A$ which is derived as a subpath of the optimal TSP tour on $A \cup B$. The same definition applies to $\lambda(B)$.]

\end{document}